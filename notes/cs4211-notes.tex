\documentclass[aspectratio=169]{beamer}
\usepackage{amsmath}
\usepackage{graphicx}
% \usepackage{oz}
\usepackage{zed-csp}
\usepackage[utf8]{inputenc}
\usepackage{newunicodechar} % For defining new Unicode characters
% The listings package supports many different programming languages
\usepackage{listings}

\usetheme{Madrid}
\definecolor{UBCblue}{rgb}{0.04706, 0.13725, 0.26667}
\usecolortheme[named=UBCblue]{structure}

% Listings Configuration
\definecolor{codegreen}{rgb}{0,0.6,0}
\definecolor{codegray}{rgb}{0.5,0.5,0.5}
\definecolor{codepurple}{rgb}{0.58,0,0.82}
\definecolor{backcolour}{rgb}{0.95,0.95,0.92}

\lstdefinestyle{mystyle}{
backgroundcolor=\color{backcolour},   
commentstyle=\color{codegreen},
keywordstyle=\color{magenta},
numberstyle=\tiny\color{codegray},
stringstyle=\color{codegreen},
basicstyle=\ttfamily\footnotesize,
breakatwhitespace=false,         
breaklines=true,                 
captionpos=b,                    
keepspaces=true,                 
numbers=left,                    
numbersep=5pt,                  
showspaces=false,                
showstringspaces=false,
showtabs=false,                  
tabsize=2
}

\lstset{style=mystyle}

\newlength{\trianglewidth}
\settowidth{\trianglewidth}{\(\vartriangleleft\)}

\newcommand{\lefttrianglebar}{%
    \mathrel{\makebox[\trianglewidth]{%
        \makebox[\trianglewidth]{\(\vartriangleleft\)}%
        \hspace*{-\trianglewidth}%
        \makebox[\trianglewidth]{\(-\)}%
    }}%
}
\newcommand{\righttrianglebar}{%
    \mathrel{\makebox[\trianglewidth]{%
        \makebox[\trianglewidth]{\(-\)}%
        \hspace*{-\trianglewidth}%
        \makebox[\trianglewidth]{\(\vartriangleright\)}%
    }}%
}

% Define the U+2987 and U+2988 character
\newunicodechar{⦇}{\ensuremath{\langle\!\langle}} % or any other substitute
\newunicodechar{⦈}{\ensuremath{\rangle\!\rangle}} % Substitute with any symbol you prefer

\title{CS4211: Formal Methods for Software Engineering}
\subtitle{Lecture Notes}
\author{Kevin Toh}
\date{}

\begin{document}

\begin{frame}
\titlepage
\end{frame}

\begin{frame}{Introduction to Formal Methods}
    \begin{itemize}
        \item Requirements are difficult to define because its written in \textbf{Natural Language} which can be imprecise and ambiguous at times.
        \item As we cannot anticipate the ways a system may be used, written test cases only covers a small subset of use cases.
        \item We want to verify if a system always satisify a certain property, in possible all cases.
        \item In Formal Methods, we use \textbf{Mathematics} to define the \textbf{structure} and \textbf{behaviour} of our software because it is \textbf{precise} and \textbf{unambiguous}
        \item Eventually, we can use a model checker to automatically verify the software by checking if a certain property holds in all cases.
    \end{itemize}
\end{frame}

\begin{frame}{The Z Specification Language}
    \begin{itemize}
        \item Based on set theory and mathematical logic
        \begin{itemize}
            \item We will be doing a recap on predicates, set theory, functions and relations next.
        \end{itemize}
        \item Uses schemas to declare object properties
        \begin{itemize}
            \item Schemas are similar to defining the structure of a class and its properties in an Object-Oriented Programming Language.
        \end{itemize}
        \item Uses operations to describe state transitions
        \begin{itemize}
            \item Each object has a state representing the values it's properties hold at a certain moment in time.
            \item Operations are similar to methods of a class. 
            \item Operations modify the state of an object.
            \item We use predicates to describe state transitions in an operation.
        \end{itemize}
        \item We can then proof that a certain property holds manually
    \end{itemize}
\end{frame}

\begin{frame}{Recap on Predicates and Logic}
    \begin{block}{Predicate}
        A statement that is either true or false.
        \begin{enumerate}
            \item There are 365 days in 2024. (\textbf{false})
            \item Let $P(x, y)$ be $x + y = 9$
            \begin{itemize}
                \item $P(4, 5)$ is \textbf{true}.
                \item $P(3, 7)$ is \textbf{false}.
            \end{itemize}
        \end{enumerate}
    \end{block}
    \begin{block}{Logic Operators}
        \begin{enumerate}
            \item Not ($\neg$) Eg: 
            \item And ($\wedge$)
            \item Or ($\vee$)
            \item Implies ($\Rightarrow$)
            \item Equivalence ($\Leftrightarrow$)
        \end{enumerate}
    \end{block}
\end{frame}

\begin{frame}{Recap on Quantifiers}
    \begin{enumerate}
        \item Universal Quantifier ($\forall$)
        \begin{itemize}
            \item Example: All natural numbers are greater than -1.
            \item Mathematically, we would write $\forall n \in \mathbb{N}, n > -1$
            \item In Z Specification, we would write $\forall n : \mathbb{N} \bullet n > -1$
            \item $\forall n : \mathbb{N} \bullet n > 0$
            \item In general, $\exists x : X \bullet P(x)$ abbreviates $P(a) \wedge P(b) \wedge P(c) \wedge \ldots$
        \end{itemize}
        \item Existential Quantifier ($\exists$)
        \begin{itemize}
            \item Example: There exists a natural number more than 0.
            \item In Z Specification, we would write $\exists n : \mathbb{N} \bullet n > 0$
            \item In general, $\exists x : X \bullet P(x)$ abbreviates $P(a) \vee P(b) \vee P(c) \vee \ldots$
        \end{itemize}
    \end{enumerate}
    \begin{block}{Differences between Mathematical Notations and Z Specification}
        \begin{itemize}
            \item In Mathemical Notation, : or $\mid$ means "such that" when used in Set Expressions.
            \item In Z Specification, : means "belongs to".
            \begin{itemize}
                \item The difference between : and $\in$ will be explained later.
            \end{itemize}
            \item In Z Specification, $\bullet$ means "such as" when writing predicates.
        \end{itemize}
    \end{block}
\end{frame}

\begin{frame}{Recap on Set Theory}
    \begin{itemize}
        \item A set is a collection of elements (or members)
        \begin{itemize}
            \item Elements are not ordered: $\{a, b, c\} = \{b, a, c\}$
            \item Elements are not repeated" $\{a, a, b\}= \{a, b\}$
            \item Given Sets
            \begin{itemize}
                \item $\mathbb{N} = \{0, 1, 2, 3, \ldots \}$ (The set of all natural numbers)
                \item $\mathbb{N}_{1} = \{1, 2, 3, \ldots \}$
                \item $\mathbb{Z} = \{0, 1, -1, 2, -2, \ldots \}$ (The set of all integers)
                \item $\mathbb{R}$ (The set of all real numbers)
                \item $\emptyset$ (Empty Set: The set with no elements)
            \end{itemize}
        \end{itemize}
        \item Membership: $x \in \mathbb{X}$ is a predicate which is
        \begin{itemize}
            \item true if x is in the set $\mathbb{X}$. Eg: $a \in \{a, b, c\}$
            \item false if x is not in the set $\mathbb{X}$. Eg: $d \in \{a, b, c\}$
        \end{itemize}
    \end{itemize}

    \begin{block}{Difference between ':' and '$\in$'}
        Example: $\forall x : \mathbb{Z} \bullet x > 5 \implies x \in \mathbb{N}$
        \begin{itemize}
            \item $x : \mathbb{Z}$ declares a new variable $x$ of type $\mathbb{Z}$
            \item $x \in \mathbb{N}$ is a predicate which is either true or false depending on the value of the declared $x$.
        \end{itemize}
    \end{block}
\end{frame}

\begin{frame}{Recap on Set Theory}
    \begin{itemize}
        \item Set Expressions
        \begin{itemize}
            \item We can express a set by listing its elements if the set is finite and small.
            \begin{itemize}
                \item $\{a, b, c, d\}$ is a finite set.
            \end{itemize}
            \item If a set is large or infinite, we can definite a set by giving a predicate which specifies precisely those elements in a set.
            \begin{itemize}
                \item $\mathbb{N}$ is an infinite set.
                \item The set of natural numbers less than 99 is $\{n : \mathbb{N} \mid n < 99\}$
                \item In general the set $\{x : \mathbb{X} \mid P(x)\}$ is the set of elements of $\mathbb{X}$ for which predicate $P$ is true.
            \end{itemize}
        \end{itemize}
        \item Set Examples
        \begin{itemize}
            \item The set of even integers is $\{z : \mathbb{Z} \mid \exists k : \mathbb{Z} \bullet z = 2k\}$
            \item The set of natural numbers which when divided by 7 leave a remaineder of 4 is $\{n : \mathbb{N} \mid \exists m : \mathbb{N} \bullet n = 7m + 4 \}$
            \item $\mathbb{N}$ is the set $\{z : \mathbb{Z} \mid z \geq 0\}$
            \item $\mathbb{N}_{1}$ is the set $\{n : \mathbb{N} \mid n \geq 1\}$
            \item If $a, b$ are any natural numbers, then $a .. b$ is defined as the set of all natural numbers between a and b inclusive.
            \begin{itemize}
                \item $a..b$ is the set $\{n : \mathbb{N} \mid a \leq n \leq b\}$
            \end{itemize}
        \end{itemize}
    \end{itemize}
\end{frame}

\begin{frame}{Recap on Set Theory}
    \begin{itemize}
        \item Subset ($\subseteq$): If $S$ and $T$ are sets, $S \subset T$ is a predicate equivalent to $\forall s : S \bullet s \in T$.
        \begin{itemize}
            \item The following predicates are true:
            \begin{itemize}
                \item $\{0, 1, 2\} \subseteq \mathbb{N}$
                \item $2..3 \subseteq 1..5$
                \item $\{a, b\} \subseteq \{a, b, c\}$
                \item $\emptyset \subseteq X$ for any set $X$
                \item $\{x\} \subseteq X \Leftrightarrow x \in X$
            \end{itemize}
        \end{itemize}
        \item Proper Subset ($\subset$): If $S$ and $T$ are sets, $S \subset T$ is a predicate equivalent to $S \subseteq T \wedge S \neq T$.
        \item Power Set ($\mathcal{P}$): If $X$ is a set, $\mathcal{P}$ $X$ (the power set of $X$) is the set of all subsets of $X$.
        \begin{itemize}
            \item $A \in$ $\mathcal{P}$ $B$ $= A \subseteq B$ 
            \item The following predicates are true:
            \begin{itemize}
                \item $\mathcal{P}\{a, b\} = \{\emptyset, \{a\}, \{b\}, \{a, b\}\}$
                \item $\mathcal{P}$ $\emptyset$ $= \{\emptyset\} \neq \emptyset$
                \item $1 .. 5 \in \mathcal{P}$ $\mathbb{N}$
                \item $2 .. 5 \in \mathcal{P}(1 .. 5)$
            \end{itemize}
            \item If $\mathbb{X}$ has $k$ elements, then $\mathcal{P}$ $X$ has $2^{k}$ elements.
        \end{itemize}
    \end{itemize}
\end{frame}

\begin{frame}{Recap on Set Theory}
    \begin{itemize}
        \item Set Operations
        \begin{itemize}
            \item Set Union: Suppose $S, T : \mathcal{P} X$ or $S \subseteq X, T \subseteq X$, then $S \cup T = \{x : X \mid x \in S \vee x \in T\}$
            \begin{itemize}
                \item $\{a, b, c\} \cup \{b, g, h\} = \{a, b, c,g, h\}$
                \item $A \cup \emptyset = A$ (for any set $A$)
            \end{itemize}
            \item Set Intersection: Suppose $S, T : \mathcal{P} X$, then $S \cap T = \{x : X \mid x \in S \wedge x \in T\}$ 
            \begin{itemize}
                \item $\{a, b\} \cap \{b, c\} = \{b\}$
                \item $\{a, b, c\} \cap \{d, g\} = \emptyset$ (disjoint sets)
                \item $A \cap \emptyset = \emptyset$ (for any set $A$)
            \end{itemize}
            \item Set Difference: Suppose $S, T : \mathcal{P} X$, then $S - T = \{x : X \mid x \in S \wedge x \not \in T\}$ 
            \begin{itemize}
                \item $\{a, b, c\} - \{b, g, h\} = \{a, c\}$
                \item $\mathbb{N}_{1} = \mathbb{N} = \{0\}$
            \end{itemize}
            \item Cartesian Product: If $A$ and $B$ are sets, then $A \times B$ is the set of all ordered pairs $(a, b)$ with $a \in A$ and $b \in B$.
            \begin{itemize}
                \item $\{a, b\} \times \{a, c\} = \{(a, a), (a, c), (b, a), (b, c)\}$
            \end{itemize}
            \item Cardinality: $\#X$ is a natural number denoting the cardinality of (number of elements in) a finite set $X$.
            \begin{itemize}
                \item $\#\{a, b, c\} = 3$
            \end{itemize}
        \end{itemize}
    \end{itemize}
\end{frame}

\begin{frame}{Recap on Relations}
    \begin{itemize}
        \item A relation $R$ from sets $A$ to $B$, is declared as $R : A \leftrightarrow B$ is a subset of $A \times B$
        \item Example: $R = \{(c, x), (c, z), (d, x), (d, y), (d, z)\}$
        \begin{itemize}
            \item The following predicates are equivalent
            \begin{enumerate}
                \item $(c, z) \in R$
                \item $c \rightarrow z \in R$
                \item $cRz$
            \end{enumerate}
        \end{itemize}
        \item \textbf{Domain:} $dom R$ is the set $\{a : A \mid \exists b : B \bullet a R b\}$
        \item \textbf{Range:} $ran R$ is the set $\{b : B \mid \exists a : A \bullet a R b\}$
    \end{itemize}
\end{frame}


\begin{frame}
    \frametitle{Types in Z Specification}
    \begin{itemize}
        \item Z specification language is \textbf{strongly typed}.
        \item Every expression is given a type.
        \item Any set can be used as a type.
        \item The following are equivalent declarations of variables $x$ and $y$ of types $A$ and $B$ respectively. 
        \begin{itemize}
            \item $(x, y) : A \times B$
            \item $x: A, y: B$
            \item $x, y : A$ (only when $B = A$)
        \end{itemize}
    \end{itemize}
\end{frame}

\begin{frame}[fragile]
\frametitle{Modelling using Z Specification}
\begin{itemize}
\item When we write a program, we can write code procedurally, functionally or in an object oriented manner.
\item Z Specification can help us model our code using two distinct sections.
\begin{enumerate}
    \item Declaration: To define variables.
    \item Predicate: Often used to define behaviours or invariants.
\end{enumerate}
\item \textbf{Example \#1:} We can define the relation \textbf{divides} between two natural numbers.

\texttt{| divides: $\mathbb{N}_1 \leftrightarrow \mathbb{N}$ \\
| ----------------------------------------- \\
| $\forall x : \mathbb{N}_1; y: \mathbb{N} \bullet x$ \underline{divides} $y \Leftrightarrow \exists k : \mathbb{N} \bullet x \cdot k = y$}

Usage: 3 \underline{divides} 6, $\lnot$ (3 \underline{divides} 7)

\item \textbf{Example \#2:} We can define the relation $\leq$ between two natural numbers.

\texttt{| $\_<=\_$: $\mathbb{N} \leftrightarrow \mathbb{N}$ \\
| ----------------------------------------- \\
| $\forall x, y: \mathbb{N} \bullet x <= y \Leftrightarrow \exists k : \mathbb{N} \bullet x + k = y$}

The relation $<=$ is the infinite subset of ordered pairs in $\mathbb{N}\times\mathbb{N}$.
\\$\{(0, 0), (0, 1), (1, 1), (0, 2) (1, 2), (2, 2), \ldots \}$
\end{itemize}
\end{frame}

\begin{frame}{Domain and Range Restriction}
\begin{itemize}
    \item Let $A, B, R, S, T$ be sets.
    \item $A$ is the domain set, $B$ is the range set and $R$ is the relation set.
    \item Note that $S$ is a subset of the domain set and $T$ is the subset of the range set.
    \item Suppose $R: A \leftrightarrow B, S \subseteq A$ and $T \subseteq B$.
    \begin{itemize}
        \item \textbf{Domain Restriction:} $S \vartriangleleft R$ is the set $\{(a, b): R \mid a \in S\}$
        \item \textbf{Range Restriction:} $R \vartriangleright T$ is the set $\{(a, b): R \mid b \in T\}$
    \end{itemize}
    \item Notice that both $S \vartriangleleft R \in A \leftrightarrow B$ and $R \vartriangleright T \in A \leftrightarrow B$, meaning that both domain restriction and range restrictions are relations from sets $A$ to $B$.
    \item If has\_sibling: People $\leftrightarrow$ People then
    \begin{itemize}
        \item female $\vartriangleleft$ has\_sibling is the relationship is\_sister\_of.
        \item has\_sibling $\vartriangleright$ female is the relationship has\_sister.
    \end{itemize}
\end{itemize}
\end{frame}

\begin{frame}{Domain and Range Subtraction}
\begin{itemize}
    \item Let $A, B, R, S, T$ be sets.
    \item $A$ is the domain set, $B$ is the range set and $R$ is the relation set.
    \item Note that $S$ is a subset of the domain set and $T$ is the subset of the range set.
    \item Suppose $R: A \leftrightarrow B, S \subseteq A$ and $T \subseteq B$.
    \begin{itemize}
        \item \textbf{Domain Subtraction:} $S \lefttrianglebar R$ is the set $\{(a, b): R \mid a \not \in S\}$
        \item \textbf{Range Subtraction:} $R \righttrianglebar T$ is the set $\{(a, b): R \mid b \not \in T\}$
    \end{itemize}
    \item The following predicates are true.
    \begin{itemize}
        \item $S \lefttrianglebar R = (A - S) \vartriangleleft R$
        \item $R \righttrianglebar T = R \vartriangleright (B - T)$
        \item $S \lefttrianglebar R \in A \leftrightarrow B$
        \item $R \righttrianglebar T \in A \leftrightarrow B$
    \end{itemize}
    \item If has\_sibling: People $\leftrightarrow$ People then
    \begin{itemize}
        \item female $\lefttrianglebar$ has\_sibling is the relationship is\_brother\_of.
        \item has\_sibling $\righttrianglebar$ female is the relationship has\_brother.
    \end{itemize}
\end{itemize}
\end{frame}

\begin{frame}{Relational Image}
\begin{itemize}
    \item Suppose the relation $R : A \leftrightarrow B$ and $S \subseteq A$
    \item $R \limg S \rimg = \{b : B \mid \exists a : S \bullet a R b\}$
    \item $R \limg S \rimg \subseteq B$
    \item Example
    \begin{itemize}
        \item $divides \limg \{8, 9\} \rimg = \{x : \mathbb{N} \mid \exists k : \mathbb{N} \bullet x = 8 \cdot k \vee 9 \cdot k \} = \{0, 8, 9, 16, 18, \ldots\}$
        \item $<= \limg {3, 7, 21} \rimg = \{x : \mathbb{N} \mid x >= 3\}$ 
    \end{itemize}
    \item In summary, the relational image returns the set of all elements $b \in B$ such that there exists an $a \in S$ with $(a, b) \in R$.
    \item The difference between relational image and range restriction is that range restriction returns the subset of $R$ which are ordered pairs of $(a, b)$ where $a \in A$ and $b \in B$ and the first element $a$ of the ordered pair is in $S$. The relational image simply just returns the set of all second elements $b$.
\end{itemize}
\end{frame}


\begin{frame}{Inverse and Relational Composition}
    \begin{itemize}
        \item \textbf{Inverse:} $R^{-1}$ is the set $\{(b, a) : B \times A \mid a R b \}$ or $R^{-1} \in B \leftrightarrow A$
        \begin{itemize}
            \item $has\_sibling^{-1} = has\_sibling$ 
            \item $divisor^{-1} = has\_divisor$ 
        \end{itemize}
        \item Example: $succ^{-1}$ = pred
        \\ \texttt{| succ: $\mathbb{N} \leftrightarrow \mathbb{N}$
        \\|--------------------------------------
        \\| $\forall x, y : \mathbb{N} \bullet x$ \underline{succ} $y \Leftrightarrow x + 1 = y$ 
        }
        \item \textbf{Relational Composition ($\comp$)}
        \begin{itemize}
            \item Suppose $R: A \leftrightarrow B$ and $S: B \leftrightarrow C$ are two relations.
            \item $R \comp S = \{(a, c) : A \times C \mid \exists b : B \mid a R b \wedge b S c\}$
            \item $R \comp S \in A \leftrightarrow C$
        \end{itemize}
        \item Examples
        \begin{itemize}
            \item $is\_parent\_of \comp is\_parent\_of = is\_grandparent\_of$
            \item $R^{0} = id[A]$
            \item $R^{1} = R$
            \item $R^{2} = R \comp R$
            \item $R^{3} = R \comp R \comp R$
        \end{itemize}
    \end{itemize}
\end{frame}

\begin{frame}{Recap on Functions}
    \begin{itemize}
        \item A (partial) function from a set $A$ to a set $B$, denoted by $f: A \pfun B$ is a subset $f$ of $A \times B$ with the property that for each $a \in A$, there is \textbf{at most one} $b \in B$ with $(a, b) \in f$.
        \item \textbf{dom f} is the set $\{a : A \mid \exists b : B \bullet (a, b) \in f\}$
        \item \textbf{ran f} is the set $\{b : B \mid \exists a : A \bullet (a, b) \in f\}$
        \item Suppose $f: A \pfun B$ and $a \in dom f$, then $f(a)$ denotes the unique image $b \in B$ that $a$ is mapped to by $f$.
        \item $(a, b) \in f$ is equivalent to $f(a) = b$
        \item \textbf{Total Function:} If the function $f: A \pfun B$ is a total function, then $f: A \rightarrow B$ if and only if $dom f = A$
    \end{itemize}
\end{frame}

\begin{frame}{Function Overriding}
    \begin{itemize}
        \item Suppose $f, g: A \pfun B$, then $f \oplus g$ is the function (dom g $\lefttrianglebar$ f) $\cup$ g.
        \item The following predicates are true:
        \begin{enumerate}
            \item dom f $\oplus$ g = dom f $\cup$ dom g
            \item a : dom g $\bullet (f \oplus g)(a) = g(a)$
            \item $\forall a :$ dom f - dom g $\bullet (f \oplus g)(a) = f(a)$
            \item f $\oplus$ g $\in a \pfun b$
        \end{enumerate}
        \item Examples
        \begin{enumerate}
            \item $\{a \rightarrow x, b \rightarrow y, c \rightarrow x\} \oplus \{a \rightarrow y\} = \{a \rightarrow y, b \rightarrow y, c \rightarrow x\}$
            \item double $\oplus$ root = $\{(0, 0), (1, 1), (2, 4), (3, 6), (4, 2), \ldots\}$ (Note: $(4, 8)$ was replaced with $(4, 2)$ as the domain 4 is both in $f$ and $g$, so the range was replaced with $2$ that was in $g$)
        \end{enumerate}
    \end{itemize}
\end{frame}

\begin{frame}{Specifying Functions}
    \begin{enumerate}
        \item Using a look-up table
        \item Declaring Axioms
        \item Using Recursion
        \item Giving an Algorithm
    \end{enumerate}
\end{frame}

\begin{frame}{Sequences}
    
\end{frame}

\end{document}