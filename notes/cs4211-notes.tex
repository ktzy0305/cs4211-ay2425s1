\documentclass[aspectratio=169]{beamer}
\usepackage{amsmath}

% The listings package supports many different programming languages
\usepackage{listings}

% Listings Configuration
\definecolor{codegreen}{rgb}{0,0.6,0}
\definecolor{codegray}{rgb}{0.5,0.5,0.5}
\definecolor{codepurple}{rgb}{0.58,0,0.82}
\definecolor{backcolour}{rgb}{0.95,0.95,0.92}

\lstdefinestyle{mystyle}{
backgroundcolor=\color{backcolour},   
commentstyle=\color{codegreen},
keywordstyle=\color{magenta},
numberstyle=\tiny\color{codegray},
stringstyle=\color{codegreen},
basicstyle=\ttfamily\footnotesize,
breakatwhitespace=false,         
breaklines=true,                 
captionpos=b,                    
keepspaces=true,                 
numbers=left,                    
numbersep=5pt,                  
showspaces=false,                
showstringspaces=false,
showtabs=false,                  
tabsize=2
}

\usetheme{Madrid}
\definecolor{UBCblue}{rgb}{0.04706, 0.13725, 0.26667}
\usecolortheme[named=UBCblue]{structure}

\title{CS4211: Formal Methods for Software Engineering}
\subtitle{Lecture Notes}
\author{Kevin Toh}
\date{}

\begin{document}

\begin{frame}
\titlepage
\end{frame}

\begin{frame}{Introduction to Formal Methods}
    \begin{itemize}
        \item Requirements are difficult to define because its written in \textbf{Natural Language} which can be imprecise and ambiguous at times.
        \item As we cannot anticipate the ways a system may be used, written test cases only covers a small subset of use cases.
        \item We want to verify if a system always satisify a certain property, in possible all cases.
        \item In Formal Methods, we use \textbf{Mathematics} to define the \textbf{structure} and \textbf{behaviour} of our software because it is \textbf{precise} and \textbf{unambiguous}
        \item Eventually, we can use a model checker to automatically verify the software by checking if a certain property holds in all cases.
    \end{itemize}
\end{frame}

\begin{frame}{The Z Specification Language}
    \begin{itemize}
        \item Based on set theory and mathematical logic
        \begin{itemize}
            \item We will be doing a recap on predicates, set theory, functions and relations next.
        \end{itemize}
        \item Uses schemas to declare object properties
        \begin{itemize}
            \item Schemas are similar to defining the structure of a class and its properties in an Object-Oriented Programming Language.
        \end{itemize}
        \item Uses operations to describe state transitions
        \begin{itemize}
            \item Each object has a state representing the values it's properties hold at a certain moment in time.
            \item Operations are similar to methods of a class. 
            \item Operations modify the state of an object.
            \item We use predicates to describe state transitions in an operation.
        \end{itemize}
        \item We can then proof that a certain property holds manually
    \end{itemize}
\end{frame}

\begin{frame}{Recap on Predicates and Logic}
    \begin{block}{Predicate}
        A statement that is either true or false.
        \begin{enumerate}
            \item There are 365 days in 2024. (\textbf{false})
            \item Let $P(x, y)$ be $x + y = 9$
            \begin{itemize}
                \item $P(4, 5)$ is \textbf{true}.
                \item $P(3, 7)$ is \textbf{false}.
            \end{itemize}
        \end{enumerate}
    \end{block}
    \begin{block}{Logic Operators}
        \begin{enumerate}
            \item Not ($\neg$) Eg: 
            \item And ($\wedge$)
            \item Or ($\vee$)
            \item Implies ($\Rightarrow$)
            \item Equivalence ($\Leftrightarrow$)
        \end{enumerate}
    \end{block}
\end{frame}

\begin{frame}{Recap on Quantifiers}
    \begin{enumerate}
        \item Universal Quantifier ($\forall$)
        \begin{itemize}
            \item Example: All natural numbers are greater than -1.
            \item Mathematically, we would write $\forall n \in \mathbb{N}, n > -1$
            \item In Z Specification, we would write $\forall n : \mathbb{N} \bullet n > -1$
            \item $\forall n : \mathbb{N} \bullet n > 0$
            \item In general, $\exists x : X \bullet P(x)$ abbreviates $P(a) \wedge P(b) \wedge P(c) \wedge \ldots$
        \end{itemize}
        \item Existential Quantifier ($\exists$)
        \begin{itemize}
            \item Example: There exists a natural number more than 0.
            \item In Z Specification, we would write $\exists n : \mathbb{N} \bullet n > 0$
            \item In general, $\exists x : X \bullet P(x)$ abbreviates $P(a) \vee P(b) \vee P(c) \vee \ldots$
        \end{itemize}
    \end{enumerate}
    \begin{block}{Differences between Mathematical Notations and Z Specification}
        \begin{itemize}
            \item In Mathemical Notation, : or $\mid$ means "such that" when used in Set Expressions.
            \item In Z Specification, : means "belongs to".
            \begin{itemize}
                \item The difference between : and $\in$ will be explained later.
            \end{itemize}
            \item In Z Specification, $\bullet$ means "such as" when writing predicates.
        \end{itemize}
    \end{block}
\end{frame}

\begin{frame}{Recap on Set Theory}
    \begin{itemize}
        \item A set is a collection of elements (or members)
        \begin{itemize}
            \item Elements are not ordered: $\{a, b, c\} = \{b, a, c\}$
            \item Elements are not repeated" $\{a, a, b\}= \{a, b\}$
            \item Given Sets
            \begin{itemize}
                \item $\mathbb{N} = \{0, 1, 2, 3, \ldots \}$ (The set of all natural numbers)
                \item $\mathbb{N}_{1} = \{1, 2, 3, \ldots \}$
                \item $\mathbb{Z} = \{0, 1, -1, 2, -2, \ldots \}$ (The set of all integers)
                \item $\mathbb{R}$ (The set of all real numbers)
                \item $\emptyset$ (Empty Set: The set with no elements)
            \end{itemize}
        \end{itemize}
        \item Membership: $x \in \mathbb{X}$ is a predicate which is
        \begin{itemize}
            \item true if x is in the set $\mathbb{X}$. Eg: $a \in \{a, b, c\}$
            \item false if x is not in the set $\mathbb{X}$. Eg: $d \in \{a, b, c\}$
        \end{itemize}
    \end{itemize}

    \begin{block}{Difference between ':' and '$\in$'}
        Example: $\forall x : \mathbb{Z} \bullet x > 5 \implies x \in \mathbb{N}$
        \begin{itemize}
            \item $x : \mathbb{Z}$ declares a new variable $x$ of type $\mathbb{Z}$
            \item $x \in \mathbb{N}$ is a predicate which is either true or false depending on the value of the declared $x$.
        \end{itemize}
    \end{block}
\end{frame}

\begin{frame}{Recap on Set Theory}
    \begin{itemize}
        \item Set Expressions
        \begin{itemize}
            \item We can express a set by listing its elements if the set is finite and small.
            \begin{itemize}
                \item $\{a, b, c, d\}$ is a finite set.
            \end{itemize}
            \item If a set is large or infinite, we can definite a set by giving a predicate which specifies precisely those elements in a set.
            \begin{itemize}
                \item $\mathbb{N}$ is an infinite set.
                \item The set of natural numbers less than 99 is $\{n : \mathbb{N} \mid n < 99\}$
                \item In general the set $\{x : \mathbb{X} \mid P(x)\}$ is the set of elements of $\mathbb{X}$ for which predicate $P$ is true.
            \end{itemize}
        \end{itemize}
        \item Set Examples
        \begin{itemize}
            \item The set of even integers is $\{z : \mathbb{Z} \mid \exists k : \mathbb{Z} \bullet z = 2k\}$
            \item The set of natural numbers which when divided by 7 leave a remaineder of 4 is $\{n : \mathbb{N} \mid \exists m : \mathbb{N} \bullet n = 7m + 4 \}$
            \item $\mathbb{N}$ is the set $\{z : \mathbb{Z} \mid z \geq 0\}$
            \item $\mathbb{N}_{1}$ is the set $\{n : \mathbb{N} \mid n \geq 1\}$
            \item If $a, b$ are any natural numbers, then $a .. b$ is defined as the set of all natural numbers between a and b inclusive.
            \begin{itemize}
                \item $a..b$ is the set $\{n : \mathbb{N} \mid a \leq n \leq b\}$
            \end{itemize}
        \end{itemize}
    \end{itemize}
\end{frame}

\begin{frame}{Recap on Set Theory}
    \begin{itemize}
        \item Subset ($\subseteq$): If $\mathbb{S}$ and $\mathbb{T}$ are sets, $\mathbb{S} \subset \mathbb{T}$ is a predicate equivalent to $\forall s : \mathbb{S} \bullet s \in \mathbb{T}$.
        \begin{itemize}
            \item The following predicates are true:
            \begin{itemize}
                \item $\{0, 1, 2\} \subseteq \mathbb{N}$
                \item $2..3 \subseteq 1..5$
                \item $\{a, b\} \subseteq \{a, b, c\}$
                \item $\emptyset \subseteq \mathbb{X}$ for any set $\mathbb{X}$
                \item $\{x\} \subseteq \mathbb{X} \Leftrightarrow x \in \mathbb{X}$
            \end{itemize}
        \end{itemize}
        \item Proper Subset ($\subset$): If $\mathbb{S}$ and $\mathbb{T}$ are sets, $\mathbb{S} \subset \mathbb{T}$ is a predicate equivalent to $\mathbb{S} \subseteq \mathbb{T} \wedge \mathbb{S} \neq \mathbb{T}$.
        \item Power Set ($\mathcal{P}$): If $\mathbb{X}$ is a set, $\mathcal{P}$ $\mathbb{X}$ (the power set of $\mathbb{X}$) is the set of all subsets of $\mathbb{X}$.
        \begin{itemize}
            \item $\mathbb{A} \in$ $\mathcal{P}$ $\mathbb{B}$ $= \mathbb{A} \subseteq \mathbb{B}$ 
            \item The following predicates are true:
            \begin{itemize}
                \item $\mathcal{P}\{a, b\} = \{\emptyset, \{a\}, \{b\}, \{a, b\}\}$
                \item $\mathcal{P}$ $\emptyset$ $= \{\emptyset\} \neq \emptyset$
                \item $1 .. 5 \in \mathcal{P}$ $\mathbb{N}$
                \item $2 .. 5 \in \mathcal{P}(1 .. 5)$
            \end{itemize}
            \item If $\mathbb{X}$ has $k$ elements, then $\mathcal{P}$ $\mathbb{X}$ has $2^{k}$ elements.
        \end{itemize}
    \end{itemize}
\end{frame}

\begin{frame}{Recap on Set Theory}
    \begin{itemize}
        \item Set Operations
        \begin{itemize}
            \item Set Union
            \item Set Intersection
            \item Set Difference
            \item Cartesian Product: If $A$ and $B$ are sets, then $A \times B$ is the set of all ordered pairs $(a, b)$ with $a \in A$ and $b \in B$.
            \begin{itemize}
                \item $\{a, b\} \times \{a, c\} = \{(a, a), (a, c), (b, a), (b, c)\}$
            \end{itemize}
            \item Cardinality: $\#X$ is a natural number denoting the cardinality of (number of elements in) a finite set $X$.
            \begin{itemize}
                \item $\#\{a, b, c\} = 3$
            \end{itemize}
        \end{itemize}
    \end{itemize}
\end{frame}

\begin{frame}{Recap on Relations}
    \begin{itemize}
        \item A relation $R$ from sets $A$ to $B$, is declared as $R : A \leftrightarrow B$ is a subset of $A \times B$
        \item Example: $R = \{(c, x), (c, z), (d, x), (d, y), (d, z)\}$
        \begin{itemize}
            \item The following predicates are equivalent
            \begin{enumerate}
                \item $(c, z) \in R$
                \item $c \rightarrow z \in R$
                \item $cRz$
            \end{enumerate}
        \end{itemize}
        \item \textbf{Domain:} $dom R$ is the set $\{a : A \mid \exists b : B \bullet a R b\}$
        \item \textbf{Range:} $ran R$ is the set $\{b : B \mid \exists a : A \bullet a R b\}$
    \end{itemize}
\end{frame}

\begin{frame}{Types in Z Specification}
    \begin{itemize}
        \item Z specification language is \textbf{strongly typed}.
        \item Every expression is given a type.
        \item Any set can be used as a type.
        \item The following are equivalent declarations of variables $x$ and $y$ of types $A$ and $B$ respectively. 
        \begin{itemize}
            \item $(x, y) : A \times B$
            \item $x: A, y: B$
            \item $x, y : A$ (only when $B = A$)
        \end{itemize}
    \end{itemize}
\end{frame}

\begin{frame}{Z Specification State Schema}
    % \begin{lstlisting}
    %     |-State------
    %     | var: type
    % \end{lstlisting}
\end{frame}

\begin{frame}{Recap on Functions}
\end{frame}

\begin{frame}{Sequences}
    
\end{frame}

\end{document}