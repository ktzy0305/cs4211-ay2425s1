\documentclass[10pt, landscape]{article}
\usepackage[scaled=0.92]{helvet}
\usepackage{calc}
\usepackage{multicol}
\usepackage[a4paper,margin=3mm,landscape]{geometry}
\usepackage{amsmath,amsthm,amsfonts,amssymb}
\usepackage{color,graphicx,overpic}
\usepackage{hyperref}
\usepackage{newtxtext} 
\usepackage{enumitem}
\usepackage[table]{xcolor}
\usepackage{mathtools}
\usepackage[normalem]{ulem}
\usepackage{graphbox}
\setlist{nosep}
% for including images
\graphicspath{ {./images/} }

\pdfinfo{
  /Title (CS4211.pdf)
  /Creator (TeX)
  /Producer (pdfTeX 1.40.0)
  /Author (Kevin Toh)
  /Subject (CS4211)
/Keywords (CS4211, nus,cheatsheet,pdf)}

% Turn off header and footer
\pagestyle{empty}

% redefine section commands to use less space
\makeatletter
\renewcommand{\section}{\@startsection{section}{1}{0mm}%
  {-1ex plus -.5ex minus -.2ex}%
  {0.5ex plus .2ex}%x
{\normalfont\large\bfseries}}
\renewcommand{\subsection}{\@startsection{subsection}{2}{0mm}%
  {-1explus -.5ex minus -.2ex}%
  {0.5ex plus .2ex}%
{\normalfont\normalsize\bfseries}}
\renewcommand{\subsubsection}{\@startsection{subsubsection}{3}{0mm}%
  {-1ex plus -.5ex minus -.2ex}%
  {1ex plus .2ex}%
{\normalfont\small\bfseries}}%
\makeatother

\renewcommand{\familydefault}{\sfdefault}
\renewcommand\rmdefault{\sfdefault}
%  makes nested numbering (e.g. 1.1.1, 1.1.2, etc)
\renewcommand{\labelenumii}{\theenumii}
\renewcommand{\theenumii}{\theenumi.\arabic{enumii}.}
\renewcommand\labelitemii{•}
\renewcommand\labelitemiii{•}

\definecolor{mathblue}{cmyk}{1,.72,0,.38}
\everymath\expandafter{\the\everymath \color{mathblue}}

% Don't print section numbers
\setcounter{secnumdepth}{0}

\setlength{\parindent}{0pt}
\setlength{\parskip}{0pt plus 0.5ex}
%% adjust spacing for all itemize/enumerate
\setlength{\leftmargini}{0.5cm}
\setlength{\leftmarginii}{0.5cm}
\setlist[itemize,1]{leftmargin=2mm,labelindent=1mm,labelsep=1mm}
\setlist[itemize,2]{leftmargin=3mm,labelindent=1mm,labelsep=1mm}
\setlist[itemize,3]{leftmargin=3mm,labelindent=1mm,labelsep=1mm}

% adding my commands
% tightcenter
\newenvironment{tightcenter}{%
  \setlength\topsep{0pt}
  \setlength\parskip{0pt}
  \begin{center}
    }{%
  \end{center}
}

% boxed
\newenvironment{tightbox}{%
  \setlength\topsep{0pt}
  \setlength\parskip{0pt}
  \begin{center}
    \begin{tabular}{|@{\hspace{\dimexpr\fboxsep+0.5\arrayrulewidth}}c@{\hspace{\dimexpr\fboxsep+0.5\arrayrulewidth}}|}
      \hline
    }
    {%
    \\ \hline
    \end{tabular}
  \end{center}
}

% fixed width box
\newenvironment{fixedbox}[1][0.7]{
  \setlength\topsep{0pt}
  \setlength\parskip{0pt}
  \begin{center}
    \begin{tabular}{|>{\centering\arraybackslash}m{#1\linewidth}|}
    \hline
  }{
  \\ \hline
  \end{tabular}
  \end{center}
}

% definition of a new term
\usepackage{soul}
\definecolor{paleyellow}{RGB}{251,243,218}
\newcommand{\definition}[2][]{\sethlcolor{paleyellow}\hl{\textbf{#2}} #1  $\rightarrow$}
% inline definition
\newcommand{\ildefinition}[1]{\sethlcolor{paleyellow}\hl{\textbf{#1}}}

% important note (attention)
\newcommand{\attention}{{\color{red}\textbf{! }}}

% nice proof
\newenvironment{niceproof}[1][Proof]
{%
  \sbox0{\textit{#1}. }%
  \list{}{\labelwidth\wd0 \leftmargin\wd0 \labelsep 0pt }
\item[\usebox0]}
  {\endlist}


\usepackage{color, soul}
\usepackage{listings}
\usepackage{inconsolata}

\definecolor{codegreen}{rgb}{0,0.6,0}
\definecolor{codegray}{rgb}{0.5,0.5,0.5}
\definecolor{codepurple}{HTML}{C42043}
\definecolor{backcolour}{HTML}{F2F2F2}
\definecolor{bookColor}{cmyk}{0,0,0,0.90}

\newcommand{\code}[1]{\texttt{\sethlcolor{backcolour}\hl{$\,$#1$\,$}}}

% SQL code blocks
% define SQL styles
\lstdefinestyle{mySQL}{%
  language=SQL,
  backgroundcolor=\color{backcolour},
  commentstyle=\color{codegreen},
  keywordstyle=\color{codepurple},
  numberstyle=\numberstyle,
  stringstyle=\color{codepurple},
  basicstyle=\scriptsize\ttfamily,
  breaklines=true,
}

\lstdefinestyle{myCode}{
backgroundcolor=\color{backcolour},   
commentstyle=\color{codegreen},
keywordstyle=\color{magenta},
numberstyle=\tiny\color{codegray},
stringstyle=\color{codegreen},
basicstyle=\ttfamily\footnotesize,
breakatwhitespace=false,         
breaklines=true,                 
captionpos=b,                    
keepspaces=true,                 
numbers=left,                    
numbersep=5pt,                  
showspaces=false,                
showstringspaces=false,
showtabs=false,                  
tabsize=2
}
\newcommand{\Mod}[1]{\ (\mathrm{mod}\ #1)}

% set lst style
\lstset{style=mySQL}

% -----------------------------------------------------------------------

\begin{document}
\raggedright
\footnotesize
\begin{multicols*}{3}
  % multicol parameters
  \setlength{\columnseprule}{0.25pt}

  \begin{center}
    \fbox{%
      \parbox{0.8\linewidth}{\centering \textcolor{black}{
          {\Large\textbf{CS4211}}
        \\ \normalsize{AY24/25 SEM 1}}
        % \\ {\footnotesize \textcolor{gray}{github/jovyntls}}
      }%
    }
  \end{center}
  \section{01. THEORY}
  \subsection{Keys}
  \begin{itemize}
    \item \definition{Superkey} A subset of attributes that uniquely identifies a tuple (all attributes) in a relation.
    \item \definition{Key} A minimal \textbf{superkey}. (Cannot be made smaller)
    \item \definition{Candidate Keys} The set of all \textbf{keys} of a given relation.
    \item \definition{Primary Key} The selected \textbf{candidate key}.
    \item \definition{Foreign Key} A subset of attributes of relation $R_1$ that refers to the \textbf{primary key} of relation $R_2$. It can contain \texttt{NULL} values.
  \end{itemize}

  \section{02. ER DIAGRAMS}
  \subsection{Cardinality Constraints}
  \begin{itemize}
    % \item Many to Many \includegraphics[scale=0.35, align=c]{03_many_to_many.png}
    % \item Many to One \includegraphics[scale=0.35, align=c]{03_many_to_one.png}
    % \item One to One \includegraphics[scale=0.35, align=c]{03_one_to_one.png}
    \item Example item
  \end{itemize}
  \subsection{Participation Constraints}
  \begin{itemize}
    \item Example item
    % \item Unconstrained \includegraphics[scale=0.4, align=c]{03_unconstrained.jpeg}
    % \item At most one (Key Constraint) \includegraphics[scale=0.4, align=c]{03_key_constrain.jpeg}
    % \item At least one (Total Participation) \includegraphics[scale=0.4, align=c]{03_total_part.jpeg}
    % \item Exactly one (Key + Total Participation) \includegraphics[scale=0.4, align=c]{03_key_total_part.jpeg}
    % \item Weak entity + Indentifying Relationship \includegraphics[scale=0.4, align=c]{03_weak_entity.jpeg}
    
  \end{itemize}

  \subsection{ISA Hierachies}
  % \includegraphics[scale=0.35]{03_isa.jpeg}
  \subsection{Aggregation}
  % \includegraphics[scale=0.35]{03_aggregate.png}
  \section{03. SQL}
  \subsection{Three-valued logic}
  \begin{itemize}
    \item SQL uses \texttt{True}, \texttt{False} and \texttt{NULL}.
    \item Suppose $x$ is \texttt{NULL}
    \begin{enumerate}
      \item \texttt{x IS NULL} $\rightarrow$ \texttt{True}
      \item \texttt{x IS NOT NULL} $\rightarrow$ \texttt{False}
      \item \texttt{x IS DISTINCT FROM y}
      \begin{itemize}
        \item $y$ is \texttt{NULL} $\rightarrow$ \texttt{False}
        \item $y$ is \textbf{not} \texttt{NULL} $\rightarrow$ \texttt{True}
      \end{itemize}
      \item \texttt{x IS NOT DISTINCT FROM y}
    \end{enumerate}
  \end{itemize}
  \begin{fixedbox}[0.9]
  Note: \texttt{NULL <> NULL} $\rightarrow$ \texttt{NULL}
  \end{fixedbox}
  \subsection{Constraints}
  \definition{Unique} \texttt{r1.column <> r2.column}
  
  \textbf{Named constraints: } \texttt{id INT CONSTRAINT nn\_id NOT NULL}

  \textbf{Foreign Key Constraints: } \texttt{ON DELETE <ACTION> / ON UPDATE <ACTION>}
  \begin{itemize}
    \item \texttt{NO ACTION}: Reject if it violates constraints
    \item \texttt{RESTRICT}: Can't be deferred
    \item \texttt{CASCADE}
    \item \texttt{SET DEFAULT}
    \item \texttt{SET NULL}
  \end{itemize}

  \textbf{Check Constraint}
  \begin{itemize}
    \item Single Column: \texttt{CHECK(hours > 0)}
    \item Multiple Columns: \texttt{CHECK(start\_year <= end\_year)}
  \end{itemize}
  \subsection{Pattern Matching}
  \begin{itemize}
    \item \texttt{\_} matches any single character.
    \item \texttt{\%} matches a sequence of 0 or more characters.
  \end{itemize}
  \subsection{Set Operations}
  \definition{Union Compatible} Two relations $R$ and $S$ are \textbf{union compatible}
  \begin{enumerate}
    \item $R$ and $S$ have the same number of attributes
    \item The corresponding attributes have the same data type / compatible domains.
  \end{enumerate}
  \begin{itemize}
    \item \texttt{Q1 UNION Q2} $\implies Q_1 \cup Q_2$
    \item \texttt{Q1 INTERSECT Q2} $\implies Q_1 \cap Q_2$
    \item \texttt{Q1 EXCEPT Q2} $\implies Q_1 - Q_2$
  \end{itemize}
  \subsection{SQL Conceptual Evaluation}
  \begin{enumerate}
    \item \texttt{FROM}
    \item \texttt{WHERE}
    \item \texttt{GROUP BY}
    \item \texttt{HAVING}
    \item \texttt{SELECT}
    \item \texttt{ORDER BY}
    \item \texttt{LIMIT / OFFSET}
  \end{enumerate}

  \section{04. RELATIONAL ALGEBRA}
  \subsection{Unary Operators}
  \begin{itemize}
    \item \textbf{Selection} $\sigma_{[c]}(R)$ Selects all tuples from a relation $R$ that satisfy the condition $c$ based on the \textbf{principle of acceptance}.
    \item \textbf{Projection} $\pi_{[\ell]}(R)$ Keeps only the columns specified in the ordered list $\ell$ and in the same order.
    \item \textbf{Renaming} $\rho_{[\mathbb{R}]}(R)$ Rename all the attributes mentioned in $\mathbb{R}$ such that for each renaming $B_i \leftarrow A_i$, the attribute $A_i$ is renamed to $B_i$.
  \end{itemize}

  \subsection{Binary Operators}
  \begin{itemize}
    \item \textbf{Union} $\cup$
    \item \textbf{Intersect} $\cap$
    \item \textbf{Except} $-$
    \item \textbf{Cross Product} $\times$
  \end{itemize}


  \subsection{SQL in RA}
  \begin{fixedbox}[0.9]
    $\pi_{[attributes]}(\sigma_{[condition](R_1 \times R_2)})$
  \end{fixedbox}
  \begin{itemize}
    \item $\pi$ Projection $\rightarrow$ \texttt{SELECT}
    \item $\times$ Cross Product $\rightarrow$ \texttt{FROM}
    \item $\sigma$ Selection $\rightarrow$ \texttt{WHERE}
  \end{itemize}

  \subsection{Joins}
  \begin{itemize}
    \item \textbf{Inner Join} $\bowtie_{[\theta]}$
    \item \textbf{Equi Join} $\bowtie_{[=]}$
    \item \textbf{Natural Join} $\bowtie$ If there is no common attributes then it is a cartesian join.
    \item \textbf{Left Outer Join}
    \item \textbf{Right Outer Join} 
    \item \textbf{Full Outer Join} 
  \end{itemize}


  \section{05. FUNCTIONS, PROCEDURES, TRIGGERS}

  \subsection{Functions}
  \begin{itemize}
    \item Function parameters can also serve as variables using the \texttt{IN} and \texttt{OUT} keyword, for example \texttt{FUNCTION swap(INOUT val1 INT, INOUT val2 INT)} and \texttt{FUNCTION sum\_to\_x(IN x INT, OUT s INT)}
    \item The \texttt{return\_type} can be an atomic value like \texttt{INTEGER}, \texttt{CHAR(10)} or a tuple such as \texttt{RECORD} or other table's names like \texttt{Scores} or even tables like \texttt{TABLE(name TEXT, mark INT, gap INT)}
  \end{itemize}
\begin{lstlisting}[language=SQL]
CREATE OR REPLACE FUNCTION func_name(val INT)
RETURNS return_type AS $$
  DECLARE
  -- local variables
  BEGIN
  -- implementation
  END;
$$ LANGUAGE plpgsql;
\end{lstlisting}

  \subsection{Procedures}
  \begin{itemize}
    \item Procedures have no return type.
    \item Example procedure usage: \texttt{CALL transfer('Alice', 'Bob', 100)}
  \end{itemize}
\begin{lstlisting}[language=SQL]
CREATE OR REPLACE PROCEDURE p_name(val INT)
AS $$
  DECLARE
  -- local variables
  BEGIN
  -- implementation
  END;
$$ LANGUAGE plpgsql;
\end{lstlisting}

  \subsection{Control Structures}
  \begin{itemize}
    \item \texttt{IF... THEN... ELSE... ENDIF}
    \item \texttt{LOOP... END LOOP}
    \item \texttt{EXIT... WHEN...}
    \item \texttt{WHILE... LOOP... END LOOP}
    \item \texttt{FOR... IN... LOOP... END LOOP}
  \end{itemize}

  \subsection{Cursors}
  Cursors enable us to access each individual row returned by a \texttt{SELECT} statement.

  Flow: Declare $\rightarrow$ Open $
  \rightarrow$ Fetch $\rightarrow$ Not Found $\rightarrow$ Close

  \subsubsection{Cursor Movement}
  \begin{itemize}
    \item \texttt{FETCH curs INTO r} $\rightarrow$ Next tuple
    \item \texttt{FETCH PRIOR FROM curs INTO r} $\rightarrow$ Prior tuple
    \item \texttt{FETCH FIRST FROM curs INTO r} $\rightarrow$ First tuple
    \item \texttt{FETCH LAST FROM curs INTO r} $\rightarrow$ Last tuple
    \item \texttt{FETCH ABSOLUTE 3 FROM curs INTO r} $\rightarrow$ 3rd tuple
  \end{itemize}

  \subsection{Triggers}
\begin{lstlisting}[language=SQL]
CREATE OR REPLACE TRIGGER func_name()
RETURNS TRIGGER AS $$
  -- implementation
$$ LANGUAGE plpgsql;

CREATE TRIGGER foo_trigger AFTER INSERT ON Table
FOR EACH ROW EXECUTE FUNCTION func_name();
\end{lstlisting}
\subsubsection{NEW and OLD keywords}

\begin{itemize}
  \item \texttt{NEW}: Insert and Update 
  \item \texttt{OLD}: Update and Delete
\end{itemize}

\subsubsection{Before and After keywords}

\begin{itemize}
  \item \texttt{BEFORE}: Executed before the \texttt{TG\_OP} 
  \item \texttt{AFTER}: Executed after the \texttt{TG\_OP} 
  \item \texttt{INSTEAD OF}: Trigger function is executed instead of \texttt{TG\_OP}
\end{itemize}

\subsubsection{Row and Statement level Triggers}
\begin{itemize}
  \item \texttt{FOR EACH ROW}: Execute trigger function for every tuple encountered. 
  \item \texttt{FOR EACH STATEMENT}: Execute trigger function once per statement. 
\end{itemize}

\subsubsection{Order of Trigger Activation}
\begin{enumerate}
  \item Before Statement Triggers
  \item Before Row Triggers
  \item After Row Triggers
  \item After Statement Triggers
\end{enumerate}

\subsubsection{Deferred Trigger}
\begin{lstlisting}[language=SQL]
CREATE CONSTRAINT TRIGGER bal_check_trigger 
AFTER INSERT OR UPDATE OR DELETE ON Account
DEFERRABLE INITIALLY DEFERRED 
FOR EACH ROW EXECUTE FUNCTION bal_check_func();
\end{lstlisting}
\begin{itemize}
  \item Deferred triggers only work for \texttt{AFTER} and \texttt{FOR EACH ROW}.
  \item Use \texttt{BEGIN TRANSACTION} and \texttt{COMMIT} to defer a deferrable trigger.
\end{itemize}


\section{06. NORMALIZATION}

  \subsubsection{Functional Dependency}
  \begin{itemize}
    \item $\{A\} \rightarrow \{B\}$ denotes that attribute $A$ uniquely decides another attribute $B$.
    \item Two objects that have the \textbf{same} $A$ value should have the \textbf{same} $B$ value.
  \end{itemize}
  
  \subsubsection{Determining FDs from requirements}
  \begin{itemize}
    \item Suppose we have a requirement \textit{no two shops should sell the \textbf{same product} to the \textbf{same customer} at the \textbf{same day} at \textbf{two different prices}}.
    \item Identify the attributes involved: \textit{ShopID}, \textit{ProductID}, \textit{CustomerID}, \textit{Date} and \textit{Price}.
    \item Create a tuple that violates the requirements.
    
    \begin{tabular}{c c c c c}
      ShopID & ProductID & CustomerID & Date & Price \\
      $S_1$ & $P_1$ & $C_1$ & 16/4/2024 & $P_1$ \\
      $S_1$ & $P_1$ & $C_1$ & 16/4/2024 & $P_2$ \\
    \end{tabular}

    \item The attribute that is \textbf{different} in this case the \textit{price} should be uniquely decided by the attributes that remain the \textbf{same} which are \textit{ShopID}, \textit{ProductID}, \textit{CustomerID} and \textit{Date}.


  \end{itemize}

  \subsubsection{Armstrong's Axioms}
  \begin{enumerate}
    \item \textbf{Reflexivity: } $\{A, B\} \rightarrow \{A\}$
    \item \textbf{Augmentation: } $\{A\} \rightarrow \{B\} \implies \{A, C\} \rightarrow \{B, C\}$
    \item \textbf{Transitivity: } $\{A\} \rightarrow \{B\}$ and $\{B\} \rightarrow \{C\}\implies \{A\} \rightarrow \{C\}$
    
  \end{enumerate}

  \subsubsection{Additional Rules}
  \begin{itemize}
    \item \textbf{Rule of Decomposition: } 
    
    $\{A\} \rightarrow \{B, C\} \implies \{A\} \rightarrow \{B\}$ and $\{A\} \rightarrow \{C\}$
    \item \textbf{Rule of Union: } $\{A\} \rightarrow \{B\}$ and $\{A\} \rightarrow \{C\}\implies \{A\} \rightarrow \{B, C\}$ 
  \end{itemize}

  \subsubsection{Computing Closures}
  Given $\{A_1, A_2, \ldots A_n\}$, the closure $\{A_1, A_2, \ldots A_n\}^{+}$ can be computed by:
  \begin{enumerate}
    \item Initialize the closure to $\{A_1, A_2, \ldots A_n\}$.
    \item If there is a FD $A_i, A_j, \ldots A_m \rightarrow B$ such that $A_i, A_j, \ldots A_m$ is in the closure, then put $B$ in the closure.
    \item Repeat step 2 until we cannot find any new attribute to put in the closure.
  \end{enumerate}
  \subsubsection{Keys, Superkeys, Prime Attributes}
  \begin{itemize}
    \item \definition{Superkey: }A set of attribute(s) in a table that decides all other attributes.
    \item \definition{Key: }A superkey that is minimal.
    \item \definition{Prime Attribute: }If an attribute appears as part of a key, it is a prime attribute.
    \begin{fixedbox}[0.95]
      Example: $\{A, B\}$ and $\{D\}$ are keys for $R(A, B, C, D)$ then the prime attributes for the table are $A$, $B$ and $D$.
    \end{fixedbox}
  \end{itemize}
  \subsubsection{Tricks to find keys}
  \begin{enumerate}
    \item Check small attribute sets first.
    \item If an attribute $A_i$ does not appear on the RHS of any FD then $A_i$ must be part of a key.
  \end{enumerate}

  \subsubsection{BCNF Definitions}
  \begin{itemize}
    \item \definition{Decomposed FD: }An FD whose \textit{right hand side} has only \textbf{one attribute}
    \item \definition{Trivial FD: }An FD whose attribute(s) on the \textit{RHS} appears on the \textit{LHS}.
    \item \definition{Non-trivial FD: }An FD whose attribute(s) on \textit{RHS} \textbf{do not appear} on \textit{LHS}. 
  \end{itemize}

  \subsubsection{BCNF}
  A table $R$ is in BCNF if \textbf{every non-trivial and decomposed FD} has a \textbf{superkey} on its \textit{left hand side}.

  \subsubsection{BCNF Simplified Check}
  Compute the closure of each attribute subset in $R$ and check if there exists a \textbf{more but not all} closure. If such a closure exists then $R$ is not in BCNF.

  \subsubsection{BCNF Decomposition}
  Input: A table $R$
  \begin{enumerate}
    \item Find a subset $X$ of attributes in $R$ such that its closure $\{X\}^{+}$ contains \textbf{more attributes} than $X$ but \textbf{does not contain all attributes} in $R$.
    \item Decompose $R$ into two tables $R_1$ and $R_2$, such that
    \begin{itemize}
      \item $R_1$ contain all attributes in $\{X\}^{+}$
      \item $R_2$ contains all attributes in $X$ as well as the attributes not in $\{X\}^{+}$.
    \end{itemize}
    \item If $R_1$ is not in BCNF, further decompose $R_1$.
    \item If $R_2$ s not in BCNF, further decompose $R_2$.
  \end{enumerate}

  \begin{fixedbox}[0.95]
    Notes:
    \begin{itemize}
      \item BCNF decomposition is a binary decomposition.
      \item The BCNF decomposition of a table may not be unique.
      \item If a table only has two attributes then it is in BCNF.
    \end{itemize}
  \end{fixedbox}

  \subsubsection{Checking if a Decomposed Table is in BCNF}
  \begin{itemize}
    \item Suppose we decomposed a table $R$ into $R_1$ and $R_2$. If we want to check of the decomposition $\delta$ is in BCNF, then we need to individually check that both $R_1$ and $R_2$ are in BCNF.
    \item Here is how to check if one decomposed table $R_i$ is in BCNF.
    
    \begin{enumerate}
      \item Enumerate the attribute subsets of $R_i$.
      \item Derive the closures of these attribute subsets on $R$, \item Project these closures onto $R_i$ by removing attributes not in $R_i$.
    \end{enumerate}
  \end{itemize}

  \subsubsection{Lossless Join Decomposition}
  \begin{itemize}
    \item When we decompose a table $R$ into $R_1$ and $R_2$, it is a \textbf{lossless-join decomposition} if the common attributes $R_1 \cap R_2$ constitute a \textbf{superkey} for either $R_1$ or $R_2$.
    \item To determine if the relations $R_i, R_j$ in a decomposition $\delta$ is a \textbf{lossless-join decomposition}, give an example, where for all $R_i, R_j \in \delta$, $R_i \cap R_j$ constitute a \textbf{superkey} for either $R_i$ or $R_j$.
    \item BCNF guarantees lossless-join decomposition.
  \end{itemize}

  \subsubsection{3NF}
  A table satisifies 3NF if and only if for every non-trivial and decomposed FD, either the \textbf{left hand side is a superkey} or the \textbf{right hand side is a prime attribute}.

  \subsubsection{Dependency Preservation}
  \begin{itemize}
    \item Let $S$ be the given set of FDs on the original table.
    \item Let $S'$ be the given set of FDs on the decomposed tables.
    \item The decomposition preserves all FDs if and only if $S$ and $S'$ are equivalent.
  \end{itemize}

  \subsubsection{Dependency Preserving Decomposition}
  Given a set of FDs $\sum$ and decomposition $\delta$:
  \begin{enumerate}
    \item Enumerate the attribute subsets of each decomposed table $R_i$ in $\delta$.
    \item Derive the closures of these attribute subsets on $R$, \item Project these closures onto $R_i$ by removing attributes not in $R_i$.
    \item Convert the closures in step 3 into non-trivial FDs and combine all the FDs for each $R_i$ into a set $S$.
    \item Check if $\sum$ and $S$ are equivalent, if not identify the FDs that are not preserved.
  \end{enumerate}


  \subsubsection{Minimal Basis}
  \begin{itemize}
    \item Let $S$ be a set of FDs
    \item The minimal basis $M$ is a set of FDs such that
    \begin{enumerate}
      \item Every FD in $S$ can be derived from $M$ and vice versa.
      \item Every FD in $M$ is a non-trivial and decomposed FD.
      \item If any FD is removed from $M$, then some FD in $S$ cannot be derived from $M$.
      \item For any FD in $M$, if we remove an attribute from its left hand side, the FD cannot be derived by $S$.
    \end{enumerate}
    \item \textbf{Minimal Basis Algorithm}
    \begin{enumerate}
      \item Transform the FDs such that each RHS \textbf{only contains one attribute.}
      \item Examine FDs which have more than one attribute on the LHS and remove redundant attributes on the LHS.
      \item Remove redundant FDs.
      \begin{itemize}
        \item Pretending to remove one FD at a time from the remaining FDs in $S$. Compute the closure of the LHS and check if the RHS is still in the closure.
        \item Check for transitivity in the existing FDs.
      \end{itemize}
    \end{enumerate}
  \end{itemize}
  \begin{fixedbox}[0.95]
    A \textbf{canonical cover} is formed by using the rule of union on FDs in the minimal basis that have the same LHS.
  \end{fixedbox}
  \subsubsection{3NF Decomposition}
  Input: A table $R$ with a set of FDs.
  \begin{enumerate}
    \item Find a minimal basis of the FDs.
    \item Combine the FDs whose left hand sides are the same.
    \item For each FD, construct a table that contains all atributes in the FD.
    \item Check if any table contains a key for $R$. If not create a table that contains a key for $R$.
    \item Remove subsumed tables.
  \end{enumerate}

\end{multicols*}
\end{document}
